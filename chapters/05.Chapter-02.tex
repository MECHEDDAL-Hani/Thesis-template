\chapter{Title Chapter Two}

\section{List}
\subsection{Unordered lists}
\begin{itemize}
  \item HTML
  \item CSS
  \item PHP
\end{itemize}

\subsection{Ordered lists}
\begin{enumerate}
  \item One.
  \item Two.
  \item Three.
\end{enumerate}

\section{Code Highlighting}
This is code example for c++.
\begin{figure}[!h]
  \begin{lstlisting}[language=c++]
    #include <iostream>
    using namespace std;
    int main() {
      cout << "Hello world!!";
      return 0;
    }
    \end{lstlisting}
  \caption{Code Example}
\end{figure}

\section{Algorithm}

\begin{algorithm}
  \caption{Algorithm Example}\label{alg:cap}
  \begin{algorithmic}
    \Require $n \geq 0$
    \Ensure $y = x^n$
    \State $y \gets 1$
    \State $X \gets x$
    \State $N \gets n$
    \While{$N \neq 0$}
    \If{$N$ is even}
    \State $X \gets X \times X$
    \State $N \gets \frac{N}{2}$  \Comment{This is a comment}
    \ElsIf{$N$ is odd}
    \State $y \gets y \times X$
    \State $N \gets N - 1$
    \EndIf
    \EndWhile
  \end{algorithmic}
\end{algorithm}

\lipsum[1-1] \ref{alg:cap}.

\section{Math Equation}

The mass-energy equivalence is described by the famous equation

\[E=mc^2\]

Discovered in 1905 by Albert Einstein.
In natural units ($c$ = 1), the formula expresses the identity

\begin{equation}
  E=mc^2
\end{equation}

\begin{displaymath}
  \sqrt{x^2+1}
\end{displaymath}


\begin{displaymath}
  \int_{a}^{b} f(x)dx = F(b) - F(a)
\end{displaymath}

\section{Math Matrix}
\begin{itemize}
  \item Example 1 :
        \par
        $\begin{matrix}
            1 & 2 & 3 \\
            a & b & c
          \end{matrix}$
        \par
  \item  Example 2 :
        \par
        $\begin{pmatrix}
            1 & 2 & 3 \\
            a & b & c
          \end{pmatrix}$
        \par
  \item  Example 3 :
        \par
        $\begin{bmatrix}
            1 & 2 & 3 \\
            a & b & c
          \end{bmatrix}$
        \par
  \item  Example 4 :
        \par
        $\begin{Bmatrix}
            1 & 2 & 3 \\
            a & b & c
          \end{Bmatrix}$
        \par
  \item  Example 5 :
        \par
        $\begin{vmatrix}
            1 & 2 & 3 \\
            a & b & c
          \end{vmatrix}$
        \par
  \item  Example 6 :
        \par
        $\begin{Vmatrix}
            1 & 2 & 3 \\
            a & b & c
          \end{Vmatrix}$
\end{itemize}

\section{Footnotes}

I'm writing to test commands.
You can insert a footnote marker \footnotemark{} . \footnotetext{Here's the footnote.}